% cGENIE QuickStartGuide document

% Andy Ridgwell, August 2014
%
% ---------------------------------------------------------------------------------------------------------------------------------
% ---------------------------------------------------------------------------------------------------------------------------------

\documentclass[10pt,twoside]{article}
%\usepackage[paper=letter,portrait=true,margin=1.5cm,ignorehead,footnotesep=1cm]{geometry}
\usepackage[margin=0.5in,ignorehead,footnotesep=1cm]{geometry} % prior \usepackage threw a geometry error...
\usepackage{graphicx}
\usepackage{hyperref}
\usepackage{paralist}
\usepackage{caption}
\usepackage{float}
\usepackage{wasysym}

\linespread{1.1}
\setlength{\pltopsep}{2.5pt}
\setlength{\plparsep}{2.5pt}
\setlength{\partopsep}{2.5pt}
\setlength{\parskip}{2.5pt}

%\addtolength{\oddsidemargin}{1.0cm}
%\addtolength{\bottommargin}{1.0cm}

\title{cGENIE Quick-start Guide: 'muffin' version [linux]}
\author{Andy Ridgwell} \date{\today} % with notes/revisions, M.N. Evans\footnote{tested on clean Debian 10.2 install, with  netcdf40.tar.gz from \href{www.seao2.info/mycgenie.html.}{cGENIE}  website installed outside of package system, in /usr/local/ per  default.}

\usepackage[normalem]{ulem}

\begin{document}

%=================================================================================================================================
%=== BEGIN DOCUMENT ==============================================================================================================
%=================================================================================================================================

\maketitle

%---------------------------------------------------------------------------------------------------------------------------------
%--- Quick-start guide for cGENIE ---------------------------------------------------------------------------------
%---------------------------------------------------------------------------------------------------------------------------------

\noindent \textbf{This is the generic Quick-start Guide for installing cGENIE-muffin on linux.}

\noindent Before you do anything, you'll need git (Google
it). Otherwise you don't get to get the code in the first place!
You'll also need to have installed or linked to an appropriate FORTRAN
compiler and netCDF library (built with the same FORTRAN
compiler). The GNU FORTRAN compiler (gfort) \textbf{version 4.4.4} or
later is recommended. It
is simplest (invariably, \texttt{simple == best}) to install netCDF
version \textbf{4.0}. This can be obtained via the
\href{http://www.unidata.ucar.edu/software/netcdf/}{UCAR website}
(and/or more Googling!) or from the
\href{http://www.seao2.info//cgenie/software/netcdf-4.0.tar.gz}{cGENIE\ website}. More
recent versions of netCDF require a little work-around, not documented
here in the QuickStart (or it would not be 'quick')\footnote{Refer to
  the User Manual.}.  \\ You are then set to go get and run the model,
which you'll do as follows:

%  and use \begin{verbatim} ./configure
%  --enable-separate-fortran \end{verbatim} so the separate netcdff is
%  produced\footnote{https://www.unidata.ucar.edu/support/help/MailArchives/netcdf/msg09420.html}.

\begin{compactenum}
\item To get a (read-only) copy of the current 'muffin' branch of
  \textit{c}GENIE source code: \\ From your home directory (or
  elsewhere, but several path variables will have to be edited -- see
  below), type:
  \vspace{-5pt}
  \begin{verbatim}
     git clone https://github.com/derpycode/cgenie.muffin.git
\end{verbatim}
    %svn co https://svn.ggy.bris.ac.uk/subversion/genie/branches/cgenie.muffin --username=genie-user cgenie.muffin this subversion respository no longer exists?
  \vspace{-5pt} for the 'head' (current development version).
%NOTE: All this must be typed continuously on ONE LINE, with a S P A C E before `\texttt{--username}', and before `\texttt{cgenie}'.
%Unless you have logged onto the \texttt{svn} server before from your computing account, you be asked for a password -- it is \texttt{g3n1e-user}.

\item You need to set a couple of environment variables -- the
  compiler name, netCDF library name, and netCDF path\footnote{If
    running using an account on one of the Bristol clusters -- the
    relevant netCDF path for each cluster appears (commented out) at
    the bottom of the file -- ensure that the appropriate value of the
    \texttt{NETCDF\_DIR} environment variable is not commented out
    (and the others are).}. These are specified in the file
  \texttt{user.mak} (\texttt{genie-main} directory).  Check that in
  \texttt{cgenie-main/makefile.arc}, under section \texttt{=== NetCDF
    paths ===}, that the lines for combined c+fortran libraries are
  uncommented, and those for separate libraries are commented.  If the
  \textit{c}genie code tree (\texttt{cgenie.muffin}) and output
  directory (\texttt{cgenie\_output}) are installed anywhere other
  than in your account HOME directory, paths specifying this will have
  to be edited in: \texttt{user.mak} and \texttt{user.sh}
  (\texttt{genie-main} directory). If using the \texttt{runmuffin*.sh}
  experiment configuration/launching scripts, you'll also have to set
  the home directory and change every occurrence of
  \texttt{cgenie.muffin} to the model directory name you are using (if
  different).

%  MNE edited user.mak for
%  \begin{verbatim}
%    GENIE_ROOT = $(HOME)/cgenie/cgenie.muffin \\
%    OUT_DIR = $(HOME)/cgenie/cgenie_output NETCDF_DIR=/usr/local/lib
%  \end{verbatim}
%  and user.sh for
%  \begin{verbatim}
%   CODEDIR=~/cgenie/cgenie.muffin OUTROOT=~/cgenie/cgenie_output
%   ARCHIVEDIR=~/cgenie/cgenie_archive
%   LOGDIR=~/cgenie/cgenie_log
%   \end{verbatim}

Installing the model code under the default directory name
(\texttt{cgenie.muffin}) in your \texttt{\$HOME} directory is hence by
far the simplest and avoids incurring additional/unnecessary pain
(configuration complexity) ...

\item To test the code installation -- change directory to
  \texttt{cgenie.muffin/genie-main} and type:
  \vspace{-5pt}
\begin{verbatim}
make testbiogem
\end{verbatim}
\vspace{-5pt}
This compiles a carbon cycle enabled configuration of \textit{c}GENIE
and runs a short test, comparing the results against those of a
pre-run experiment (also downloaded alongside the model source
code). It serves to check that you have the software environment
correctly configured. If you are unsuccessful here ... double-check
the software and directory environment settings in \texttt{user.mak}
(or \texttt{user.sh}) and for a netCDF error, check the value of the
\texttt{NETCDF\_DIR} environment variable. (Refer to the User Manual
for addition fault-finding tips.) If environment variables are
changed: before re-trying the test, you will need to type:
\vspace{-5pt}\begin{verbatim}
make cleanall
\end{verbatim}\vspace{-5pt}

\end{compactenum}

% MNE notes on make testbiogem: error: /usr/bin/ld: cannot find
% -lnetcdff, true enough.  in the INSTALL file for netcdf-4.0, it says
% ''If you have Fortran 77 or Fortran 90 compilers, then the Fortran
% library will also be built (libnetcdff.a).''  Do we have fortran 90
% compiler? According to here:
% https://www.unidata.ucar.edu/support/help/MailArchives/netcdf/msg09420.html
% you need to ./configure --enable-separate-fortran to get a separate
% libnetcdff which is required in the cgenie compile.  Trying that -
% this gets cgenie compiled.  Or as in the cygwin guide, make sure the
% netcdf path in makefile.arc is set to compile combined c+fortran
% library.

\noindent That is it for the basic installation. To run the model it
is a simple matter of calling the '\texttt{runmuffin.sh}' shell script
from \texttt{genie-main} and supplying a couple of parameter values,
e.g.:
\vspace{-5pt}\small
\begin{verbatim}
./runmuffin.sh cgenie.eb_go_gs_ac_bg.worjh2.ANTH / EXAMPLE.worjh2.Caoetal2009.SPIN 10000
\end{verbatim}\normalsize\vspace{-5pt} Refer to the
\textit{c}GENIE \texttt{User\_manual} for more information regarding
installing, running, and analyzing model output, and \textit{c}GENIE
\texttt{Examples} for more information on this specific
example.\footnote{latex source for all the documents can be found in
  the \texttt{genie-docs} directory, with recent PDF versions at the
  \href{www.seao2.info/mycgenie.html.}{cGENIE} website. \uline{Read
    the \textit{c}GENIE \texttt{READ-ME}}}.

  % MNE Notes on using runmuffin.sh:
  % I set HOMEDIR=$HOME/cgenie/ as for user.sh and user.mak
  
%=================================================================================================================================
%=== END DOCUMENT ================================================================================================================
%=================================================================================================================================

\end{document}
