% cGENIE QuickStartGuide document

% Andy Ridgwell, August 2014
%
% ---------------------------------------------------------------------------------------------------------------------------------
% ---------------------------------------------------------------------------------------------------------------------------------

\documentclass[10pt,twoside]{article}
\usepackage[paper=a4paper,portrait=true,margin=1.5cm,ignorehead,footnotesep=1cm]{geometry}
\usepackage{graphicx}
\usepackage{hyperref}
\usepackage{paralist}
\usepackage{caption}
\usepackage{float}
\usepackage{wasysym}
\usepackage{enumitem}

\linespread{1.1}
\setlength{\pltopsep}{2.5pt}
\setlength{\plparsep}{2.5pt}
\setlength{\partopsep}{2.5pt}
\setlength{\parskip}{2.5pt}

%\addtolength{\oddsidemargin}{1.0cm}
%\addtolength{\bottommargin}{1.0cm}

\title{cGENIE Quick-start Guide: 'muffin' version [UoB cluster]}
\author{Andy Ridgwell}
\date{\today}
\usepackage[normalem]{ulem}

\begin{document}

%=================================================================================================================================
%=== BEGIN DOCUMENT ==============================================================================================================
%=================================================================================================================================

\maketitle

%---------------------------------------------------------------------------------------------------------------------------------
%--- Quick-start guide for cGENIE ---------------------------------------------------------------------------------
%---------------------------------------------------------------------------------------------------------------------------------

\noindent \textbf{This is the Quick-start Guide for installing cGENIE-muffin on a University of Bristol GENIE cluster (one of: \texttt{iwan}, \texttt{sprout}, \texttt{almond}). For installing on any other platform/machine, refer to the appropriate specific Quick-start Guide:}
\vspace{-10pt}
\begin{itemize}[noitemsep]
\item[] \texttt{cGENIE.muffin.QuickStartGuide-linux.pdf}
\item[] \texttt{cGENIE.muffin.QuickStartGuide-Mac.pdf}
\item[] \texttt{cGENIE.muffin.QuickStartGuide-Windows.pdf}
\end{itemize}

\noindent To get hold of, configure (if necessary), and test the cGENIE ('muffin') model code:

\begin{compactenum}
\item Log in (d'uh!).
\item Download a (read-only) copy of the current 'muffin' branch of \textit{c}GENIE source code, as follows:
\\ From your home directory (or elsewhere, but several path variables will then have to be edited ...), type:
\vspace{-5pt}\begin{verbatim}
svn co https://svn.ggy.bris.ac.uk/subversion/genie/branches/cgenie.muffin
--username=genie-user cgenie.muffin
\end{verbatim}\vspace{-5pt}
for the 'head' (current development version).
NOTE: All this must be typed continuously on ONE LINE, with a S P A C E before `\texttt{--username}', and before `\texttt{cgenie}'.
Unless you have logged onto the \texttt{svn} server before from this particular computing account, you be asked for a password -- it is \texttt{g3n1e-user}.

\item   You need to check (and may need to change) an environment variable that needs to be set specific to the cluster you are using -- in the file \texttt{user.mak} (in the \texttt{cgenie.muffin/genie-main} directory) -- edit the netCDF path (under \texttt{\# === NetCDF library ===}) by un-commenting (delete the \texttt{\#} character) the line for the cluster name you are using, and making sure all other \texttt{NETCDF\_DIR} path values are commented out (add a \texttt{\#} character to the start of the line).

\item   Test the code installation (just in case). To do this:

Change directory to \texttt{cgenie.muffin/genie-main} and type:
\vspace{-5pt}\begin{verbatim}
make testbiogem
\end{verbatim}\vspace{-5pt}
This compiles a carbon cycle enabled configuration of \textit{c}GENIE and runs a short test, comparing the results against those of a pre-run experiment (also downloaded alongside the model source code). It serves to check that you have the software environment correctly configured. If you are unsuccessful here ... too bad (try the svn code check-out again, also check that you are doing this from your home directory).

\end{compactenum}

\noindent That is is for the basic installation. To run the model it is a simple matter of calling the  '\texttt{runmuffin.sh}'  shell script from \texttt{genie-main} and supplying a couple of parameter values, e.g.:
\vspace{-5pt}\small\begin{verbatim}./runmuffin.sh cgenie.eb_go_gs_ac_bg.worjh2.ANTH / EXAMPLE.worjh2.Caoetal2009.SPIN 10000\end{verbatim}\normalsize\vspace{-5pt}
Refer to the \textit{c}GENIE \texttt{User\_manual} for more information regarding installing, running, and analyzing model output, and \textit{c}GENIE \texttt{Examples} for more information on this specific example.\footnote{latex source for all the documents can be found in the \texttt{genie-docs} directory, with recent PDF versions at www.seao2.info/mycgenie.html.} \uline{Read the \textit{c}GENIE \texttt{READ-ME}}.

%=================================================================================================================================
%=== END DOCUMENT ================================================================================================================
%=================================================================================================================================

\end{document}
